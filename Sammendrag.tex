\section{Abstrakt}

    Denne rapporten tar for seg to forsøk  som ble gjennomdørt under laboratorieøving 3 i TFE4101. Det første forsøket gikk ut på å gjennomføre målinger av en RC-krets ved hjelp av oscilloskop.
    Med hensikt å analysere en kondensators effekt på et AC signal.
    Det andre forsøket gikk ut på å analysere en CMOS-krets med forskjellige inngangssignaler;
    CMOS er en innfallsvinkel for å utvikle digitale kombinatoriske kretser. Hensikten med det andre forsøket vundersøke de elektriske egenskapene til en CMOS-krets

    Kondensatoren i RC-Kretsen hadde en dempende effekt på den oscillerende spenningen fra signal generatoren. Det vil kunne være mulig å annvende dette til å stabilisere et analogt signal for å fjerne støy. En annen mulig annvendelse er en AC til DC transformator

    Spenningen i CMOS kretsen varierte selv for signaler som skal tolkes som samme binære verdi. En modell som annser logiske porter som perfekte matematiske funksjoner vil være en grov approksimasjon og kan i enkelte tilfeller medføre feile estimasjoner. Dette kommer fra at logiske kretser i bunn består av elektriske komponenter, som ikke er matematisk perfekte, uten motstand, og upåvirket av utvendige faktorer. 