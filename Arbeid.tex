\section{Gjennomføring}

Gjennomføringen deles i to hoveddeler, RC-Krets og CMOS logikk. De to arbeidene er uavhengige av hverandre.\\
Ved porter refereres det til inngangene på veroboard sokkel, som samsvarer med de nederst på kretskort figuren i appendiks 1

\subsection{RC-Krets}

\subsubsection{Lodding}
Først ble kretsen, som beskrevet i Figur \ref{fig:rc}, loddet på kretskortet. En $1k\Omega$ motstand ble loddet på posisjon $R_{10}$, og en 15nF kondensator ble loddet på posisjon $C_{1}$

\begin{figure}[!htb]
    \centering
    \begin{circuitikz}
        \draw
            (0,0) to [vsourcesquare] (0,6)
            to [european resistor] (12,6)
            to [capacitor] (12,0) -- (0,0)
        ;
        \draw[fill] (0,6) circle [radius=0.1];
        \draw[fill] (12,6) circle [radius=0.1];
        \draw[fill] (12,0) circle [radius=0.1];
        \draw[fill] (0,0) circle [radius=0.1];
        \node [above] at (6,5) {$R_{10}=1k\Omega$};
        \node [above left] at (12,2) {$C_{1}=1nF$};
        \node [above left] at (0, 2) {$Sig.gen=(0,4)V$};
    \end{circuitikz}
    \caption{RC-Krets}
    \label{fig:rc}
\end{figure}

\subsubsection{Signalgenerator}
Signalgeneratoren ble stilt inn slik at den sendte ut en firkantpuls med frekvens på 3 KHz og amplitude på 4V, med 2V i offset slik at signalet har toppunkt i 4V og bunnpunkt i 0V.

\subsubsection{Måling}
Signalgeneratoren, ved hjelp av et T-ledd, ble så koblet til både kretsen via veroboardet (port 19 og port 17), og til oscilloskopet. Ved å måle spenningen over kondensatoren vc (port 18 og port 17) og spenningen fra generatoren på forskjellige kanaler var det mulig å se både spenningen som signalgeneratoren påtrykte, og hvordan spenningen endret seg etter å ha gått gjennom kondensatoren.

\subsubsection{Sitering}
I dette dokumentet har jeg valgt å bruke Biblatex. Det er flere forskjellige pakker tilgjengelig for
kildehenvisning, men det er denne jeg er mest kjent med. For å sitere en kilde er dere nødt til å gjøre 4 ting:
\begin{enumerate}
    \item definere referansestil og referansebibliotekfil (gjort i setup.tex).
    \item lage selve kilden i en .bib-fil.
    \item referere ved å benytte cite-kommandoen.
    \item printe referanseliste med printbibliography-kommandoen (gjort i main.tex).
\end{enumerate}

I mylib.bib finner dere noen eksempler på kilder. Ellers kan jeg anbefale dokumentasjonen for biblatex \cite{biblatex}.
Nå som vi har referert til en kilde legges denne til i lista. Der havner de i rekkefølgen de ble referert til.
Legg spesielt merke til hvordan man kun kildene som faktisk referes til vises. 

%Hvis dere fjerner kommentaren på denne kilden~\cite{airsim_ikke_i_bruk}, vil den også vises i lista.

\newpage % bruk denne for å starte en ny side.
\subsection{CMOS logikk}

\subsubsection{Lodding}
Først ble kretsen, som beskrevet i Figur \ref{fig:lg}, loddet opp på kretskortet, med verdier som anvist i figuren. I tillegg måtte strappingen S2 loddes inn for koblet sammen logikk kretsen og driver kretsen.


\begin{figure}[!htb]
    \centering
    \begin{circuitikz}
        % Draw components
        \draw (0,2) node[circ, scale=1.5] (A){A};
        \draw (0,1) node[circ, scale=1.5] (B){B};
        \draw (0,0) node[circ, scale=1.5] (C){C};

        \draw (1, -2) node[genericshape, rotate=90] (R4){$R_4=1.5M\Omega$};
        \draw (2, -2) node[genericshape, rotate=90] (R3){$R_3=1.5M\Omega$};
        \draw (3, -2) node[genericshape, rotate=90] (R2){$R_2=1.5M\Omega$};
        \draw (8, -2) node[genericshape, rotate=90] (R5){$R_5=4.7k\Omega$};
        \draw (9, 2) node[genericshape] (R6){$R_6=4.7k\Omega$};
        \draw (12, -2) node[genericshape, rotate=90] (R7){$R_7=4.7k\Omega$};

        \draw (1,0) node[circ, scale=1.5] (R4P){};
        \draw (2,1) node[circ, scale=1.5] (R3P){};
        \draw (3,2) node[circ, scale=1.5] (R2P){};
        \draw (2, -3) node[circ, scale=1.5] (JP){};
        \draw (8, 2) node[circ, scale=1.5] (GP){};
        
        \draw (2, -4) node[ground] (G){$V_{SS} - Jord$};
        \draw (8, -4) node[ground] (G2){$V_{SS} - Jord$};
        \draw (12, -5) node[ground] (G3){$V_{SS} - Jord$};

        \draw (12, 4) node[vcc] (vdd){$V_{DD}$};

        \draw (5,1) node[nand port] (nand1){};
        \draw (7,2) node[nand port] (nand2){};

        \draw (12, 2) node[npn] (npn){BC547C};

        \draw (12, -4) node[emptydiodeshape, rotate=-90] (diode){Lysdiode};

        % Draw lines
        \draw (C) -| (nand1.in 2);
        \draw (B) -| (nand1.in 1);
        \draw (A) -| (nand2.in 1);
        \draw (G) -- (JP);
        \draw (G2) -- (R5);
        \draw (R5) -- (GP);
        \draw (nand2.out) -- (GP);
        \draw (GP) -- (R6);
        \draw (R6) -- (npn);
        \draw (JP) -| (R4);
        \draw (JP) -| (R3);
        \draw (JP) -| (R2);
        \draw (R4) -- (R4P);
        \draw (R3) -- (R3P);
        \draw (R2) -- (R2P);
        \draw (npn.E) -- (R7);
        \draw (npn.C) -- (vdd);
        \draw (R7) -- (diode);
        \draw (diode) -- (G3);
        \draw (nand1.out) -| (nand2.in 2);
    \end{circuitikz}
    \caption{Logisk-Krets}
    \label{fig:lg}
\end{figure}

\subsubsection{Måling}
Det ble så målt spenningene til nodene D og Q på figur 2 med forskjellige inn-signaler på A, B, og C. 
Spenningene ble målt i fra jord (VSS). For å måle spenningen til node D ble proben til oscilloskop satt på port 9 (VSS) og port 4 på veroboard sokkelen. 
For node Q ble port 9 og port 5 brukt.