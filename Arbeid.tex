\section{Gjennomføring}

Gjennomføringen deles i to hoveddeler, RC-Krets og CMOS logikk. De to arbeidene er uavhengige av hverandre.\\
Ved porter refereres det til inngangene på veroboard sokkel, som samsvarer med de nederst på kretskort figuren i appendiks 1

\subsection{RC-Krets}

\subsubsection{Lodding}
Først ble kretsen, som beskrevet i Figur~\ref{fig:krets_3_3}, loddet på kretskortet. En 1k$\Omega$ motstand ble loddet på posisjon $R_{10}$, og en 15nF kondensator ble loddet på posisjon $C_{1}$

\begin{figure}[!htb]
    \centering
    \begin{circuitikz}
        \draw
            (0,0) to [vsourcesquare] (0,6)
            to [european resistor] (12,6)
            to [capacitor] (12,0) -- (0,0)
        ;
        \draw[fill] (0,6) circle [radius=0.1];
        \draw[fill] (12,6) circle [radius=0.1];
        \draw[fill] (12,0) circle [radius=0.1];
        \draw[fill] (0,0) circle [radius=0.1];
        \node [above] at (6,5) {$R_{10}=1k\Omega$};
        \node [above left] at (12,2) {$C_{1}=1nF$};
        \node [above left] at (0, 2) {$Sig.gen=(0,4)V$};
    \end{circuitikz}
    \caption{RC Krets}
    \label{fig:krets_3_3}
\end{figure}

\subsubsection{Signalgenerator}
Signalgeneratoren ble stilt inn slik at den sendte ut en firkantpuls med frekvens på 3 KHz og amplitude på 4V, med 2V i offset slik at signalet har toppunkt i 4V og bunnpunkt i 0V.

\subsubsection{Måling}
Signalgeneratoren, ved hjelp av et T-ledd, ble så koblet til både kretsen via veroboardet (port 19 og port 17), og til oscilloskopet. Ved å måle spenningen over kondensatoren vc (port 18 og port 17) og spenningen fra generatoren på forskjellige kanaler var det mulig å se både spenningen som signalgeneratoren påtrykte, og hvordan spenningen endret seg etter å ha gått gjennom kondensatoren.

\subsubsection{Sitering}
I dette dokumentet har jeg valgt å bruke Biblatex. Det er flere forskjellige pakker tilgjengelig for
kildehenvisning, men det er denne jeg er mest kjent med. For å sitere en kilde er dere nødt til å gjøre 4 ting:
\begin{enumerate}
    \item definere referansestil og referansebibliotekfil (gjort i setup.tex).
    \item lage selve kilden i en .bib-fil.
    \item referere ved å benytte cite-kommandoen.
    \item printe referanseliste med printbibliography-kommandoen (gjort i main.tex).
\end{enumerate}

I mylib.bib finner dere noen eksempler på kilder. Ellers kan jeg anbefale dokumentasjonen for biblatex \cite{biblatex}.
Nå som vi har referert til en kilde legges denne til i lista. Der havner de i rekkefølgen de ble referert til.
Legg spesielt merke til hvordan man kun kildene som faktisk referes til vises. 

%Hvis dere fjerner kommentaren på denne kilden~\cite{airsim_ikke_i_bruk}, vil den også vises i lista.

\newpage % bruk denne for å starte en ny side.
\subsection{CMOS logikk}

\subsubsection{Lodding}
Først ble kretsen, som beskrevet i figur 2 (3-3), loddet opp på kretskortet, med verdier som anvist i figuren. I tillegg måtte strappingen S2 loddes inn for koblet sammen logikk kretsen og driver kretsen.


\begin{figure}[!htb]
    \centering
    \begin{circuitikz}
        \draw (0,4) node[circ, scale=1.5] (A){A};
        \draw (0,2) node[circ, scale=1.5] (B){B};
        \draw (0,0) node[circ, scale=1.5] (C){C};

        \draw (2, -4) node[genericshape, rotate=90] (R4){$R_4$};

        \draw (2,0) node[circ, scale=1.5] (R4P){};
        \draw (4,2) node[circ, scale=1.5] (R3P){};
        \draw (6,4) node[circ, scale=1.5] (R2P){};

        \draw (10,1) node[and port] (and1){};
        \draw (12,3) node[and port] (and2){};

        \draw (C) -| (and1.in 2);
        \draw (B) -| (and1.in 1);
        \draw (A) -| (and2.in 1);
        \draw (R4) -- (R4P);
        \draw (and1.out) -| (and2.in 2);
    \end{circuitikz}
    \caption{Logisk Krets}
    \label{fig:krets_3_3}
\end{figure}

\subsubsection{Måling}
Det ble så målt spenningene til nodene D og Q på figur 2 med forskjellige inn-signaler på A, B, og C. 
Spenningene ble målt i fra jord (VSS). For å måle spenningen til node D ble proben til oscilloskop satt på port 9 (VSS) og port 4 på veroboard sokkelen. 
For node Q ble port 9 og port 5 brukt.



\subsection{tikz}
Tikz er et veldig kraftig tegneverktøy til Latex, og denne innføringen skraper bare såvidt på overfraten.
I denne pakken er det så mange muligheter at jeg ikke kommer til å skrive noe detaljert rundt det.
For en introduksjon til grunnpakken av tikz, kan dere se på Jaques Crémer sin introduksjon \cite{tikz_intro}.

 viser en enkel kretstegning tegnet med circuitikz-pakken, mens figur~\ref{fig:arb_3D_fig} er vesentlig mye mer avansert, og benytter seg av tikz-3dplots-pakken.


% --------------------------------------------------------------------------
% Dette er kun for de som er interessert i mer avansert figurlaging med Tikz
% --------------------------------------------------------------------------

% \begin{figure}[!htb]
%     \centering
%     \tdplotsetmaincoords{70}{100}
%     \begin{tikzpicture}[tdplot_main_coords, scale = 0.7]
%         \coordinate (O) at (0,0,0);
        
%         \pgfmathsetmacro{\rvec}{4}
%         \pgfmathsetmacro{\thetavec}{120}
%         \pgfmathsetmacro{\phivec}{50}
        
%         \tdplotsetcoord{P}{\rvec}{\phivec}{\thetavec}
        
%         \draw[thick, ->] (-3,0,0) -- (5,0,0) node [below left]{$x$};
%         \draw[thick, ->] (0,-1,0) -- (0,5,0) node [below right]{$y$};
%         \draw[thick, ->] (0,0,-1) -- (0,0,5) node [below left]{$z$};
        
%         \draw[->, thick, red] (O) -- (P);
%         \draw[] (P) node[above right]{$P$};
%         \node at (P){\textbullet};
%         \draw[dashed, thick, red] (O) -- (Pxy);
%         \draw[dashed, thick, blue] (Pxy) -- (0,2.654,0); %rsin(theta)sin(phi)
%         \draw[dashed, thick, blue] (Pxy) -- (-1.286,0,0); %rcos(theta)sin(phi)
%         \draw[dashed, thick, green] (P) -- (0,0,2.571); %rcos(phi)
        
%         \tdplotdrawarc{(O)}{1}{0}{\thetavec}{anchor=north}{$\theta$}
%         \tdplotsetthetaplanecoords{\thetavec}
%         \tdplotdrawarc[tdplot_rotated_coords]{(O)}{1}{0}{\phivec}{anchor=south west}{$\phi$}
        
%         \foreach \angle in {0,60,...,300} {
%             \tdplotsetthetaplanecoords{\angle}
%             \ifthenelse{\angle=\thetavec}{
%                 \draw[tdplot_rotated_coords] (\rvec,0,0) arc (0:180:\rvec);
%             }{
%                 \ifthenelse{\angle<90 \OR \angle>270}{
%                     \draw[tdplot_rotated_coords] (\rvec,0,0) arc (0:180:\rvec);
%                 }{
%                     \draw[dash pattern=on 3pt off 10pt,tdplot_rotated_coords] (\rvec,0,0) arc (0:180:\rvec);
%                 }
%             }
%         } % end \foreach
        
%         \draw[] (0,-\rvec,0) arc (-90:120:\rvec);
%         \draw[dashed] (120:\rvec) arc (120:270:\rvec);
        
%         \foreach \angle in {-60,-59.5,...,0} {
%             \tdplotsetthetaplanecoords{\angle}
%             \draw[tdplot_rotated_coords, color = gray!60, opacity = 0.7] (0:\rvec) arc (0:180:\rvec);
%         } % end foreach
        
%         \shade[ball color = blue!40, opacity = 0.4] (0,0) circle (4cm); %cm does fix things :O
        
%         \tdplotsetrotatedcoordsorigin{(P)}
%         \tdplotsetrotatedcoords{0}{-90}{90}
%         \draw[tdplot_rotated_coords,->] (-1,0,0) -- (3,0,0) node [below right]{$x'$};
%         \draw[tdplot_rotated_coords,->] (0,-1,0) -- (0,3,0) node [right]{$y'$};
%         \draw[tdplot_rotated_coords,->] (0,0,-1) -- (0,0,3) node [above right]{$z'$};
    
%     \end{tikzpicture}
    
    
%     \caption{Avansert 3D-figur}
%     \label{fig:arb_3D_fig}
% \end{figure}