% LaTeX hjelpedokument for TFE4101
% Laget av Sveinung Haugane, vår 2019.

\documentclass[12pt, a4paper]{article}

% Pakker inkluderes ved å benytte \usepackage[<package options>]{<package name>}.
% Dere kommer ikke unna bruk av pakker, hvis dere ønsker å bruke LaTeX effektivt.
% Pakken geometry gir mange fler valg for å sette marger og sidelayout.
\usepackage[lmargin=25mm,rmargin=25mm,tmargin=27mm,bmargin=30mm]{geometry}

% Bestemmer hvilken enkoding som benyttes for input fra keyboard. Anbefaler utf8.
% Dere vil da mest sannsynlig ikke få problemer med æøå.
\usepackage[utf8]{inputenc}

% Fontenc styrer fonten som blir brukt. T1 fungerer fint.
\usepackage[T1]{fontenc}

% Sett inn ekstern fil i hoveddokument.
\input{setup.tex}

% De to linjene nedenfor sikrer at nye avsnitt ikke starter med tab, og at det blir 
% mellomrom mellom avsnittene. Dere skal få styre dette selv, men personlig liker jeg
% innstillingen nedenfor.
\setlength\parindent{0pt}
\setlength{\parskip}{1em}

% Hele dokumentets innhold defineres i rekkefølge mellom \begin{document} og 
% \end{document} (titlepage og noen andre ting er spesielle tilfeller)
\begin{document}

% Det som skal på rapportens forside kan defineres innenfor et titlepage scope.
% Linjeskift kan tvinges frem med dobbel backspace \\ eller tom kodelinje.
% Titlepage blir ikke nummerert.
\begin{titlepage}
    % Tekst i mellom \begin{center} og \end{center} blir sentrert. 
    \begin{center}
        % \vspace{<lengde>} setter hvor mye whitespace som skal være i mellom linjene. 
        % For å sette mellomrom relativt til marg, må man benytte \vspace*{<lengde>}
        \vspace*{1.5cm}
        % Sjekk denne for å se på tekststørrelser: https://texblog.org/2012/08/29/changing-the-font-size-in-latex/
        {\Huge\textbf{Labrapport Lab 3: RC-Krets og CMOS logikk}}\\
        \vspace{1cm}
        {\Large av Arran Kostveit Gabriel og Stefan Mack}\\
        \vspace{1cm}
        \today\\ % Setter inn dagens dato ved kompilering
        \vspace{1cm}
        Labgruppe 24, pulje 3\\
        
        %\vfill fyller siden med whitespace, slik at teksten som følger havner neders.
        %\hfill gjør tilsvarende horisontalt.
        \vfill
        {
            \large TFE4101 Krets og Digitalteknikk\\ 
            Norges Teknisk-Naturvitenskapelige Universitet
        }
        
        
        
    \end{center}
\end{titlepage}

% \pagenumbering velger automatisk sidenummereringsstil.
%   - gobble (ingen nummerering)
%   - arabic (normal nummerering)
%   - roman (romerske tall)
\pagenumbering{roman}

% \Clearpage sier at det som kommer på de følgende linjene skal starte på neste side.
\clearpage

% For oversikten sin del kan man med fordel dele opp teksten i flere underdokumenter.
% For å sette et dokument inn i hoveddokumentet brukes \input{</path/to/filnavn.tex>}
% Hvis det ligger i samme mappe som hoveddokumentet trengs ingen spesiell path.
\section{Sammendrag}

    Denne rapporten tar for seg to forsøk  som ble gjennomdørt under laboratorieøving 3 i TFE4101. Det første forsøket gikk ut på å gjennomføre målinger av en RC-krets ved hjelp av oscilloskop.
    Med hensikt å analysere en kondensators effekt på et AC signal.
    Det andre forsøket gikk ut på å analysere en CMOS-krets med forskjellige inngangssignaler;
    CMOS er en innfallsvinkel for å utvikle digitale kombinatoriske kretser. Hensikten med det andre forsøket vundersøke de elektriske egenskapene til en CMOS-krets.

    Kondensatoren i RC-Kretsen hadde en dempende effekt på den oscillerende spenningen fra signalgeneratoren. Det vil kunne være mulig å annvende dette til å stabilisere et analogt signal for å fjerne støy. En annen mulig annvendelse er en AC til DC transformator.

    Spenningen i CMOS kretsen varierte selv for signaler som skal tolkes som samme binære verdi. En modell som annser logiske porter som perfekte matematiske funksjoner vil være en grov approksimasjon og kan i enkelte tilfeller medføre gale estimater. Dette kommer fra at logiske kretser i bunn består av elektriske komponenter, som ikke er matematisk perfekte, uten motstand, og upåvirket av utvendige faktorer. \clearpage

% \tableofcontents setter inn automatisk generert innholdsfortegnelse.
% Hva som vises her er bestemt av kapitteloverskriftene. Kommer tilbake til dette.
\tableofcontents
\addcontentsline{toc}{section}{Innhold}
\clearpage
\pagenumbering{arabic}
\setcounter{page}{1}

\section{Teori}

Denne labben bestod av to forsøk, første del gikk ut på å gjøre målinger av en RC-krets ved hjelp av oscilloskop.
Med hensikt å analysere en kondensators effekt på et AC signal.
Andre del gikk ut på å analysere en CMOS krets med forskjellige inngangssignaler;
CMOS er en innfallsvinkel for å utvikle digitale kombinatoriske kretser.

\subsection{Teori del 1.}

    \begin{itemize}
        \item[-] Kondensator: En kondensator er en passiv elektrisk komponent viss bruksområde er å lagre elektrisk ladning.
        Den består av to elektroder skilt med et dielektrisk materiale.
        Et dielektrisk materiale har elektrisk isolerende egenskaper og polariseres når det påvirkes av et elektrisk felt.
        Kondensatorer beskrives symbolsk ved uttrykket $F = \frac{Q}{U}$, der $Q$ er ladning målt i Coulomb, $U$ er spenning målt i Volt og $F$ er kapasitansen målt i Fahrrad.
        Fahrrad tilsvarer altså Coulomb/Volt.
        Kapasitansen beskriver mengden elektroner som må tilføres elektrodene for å danne en gitt spenning over dem.
        \item[-] RC-Krets: En enkel RC-krets består av en motstand og en kondensator koblet i serie se figur (3-1 ish).
        Spenning over kondensatoren i en slik krets kan beskrives ved $v_{C}(t) = V_{kilde} \cdot \left( 1 - e^{-\frac{t}{\tau}} \right)$ under oppladning og $v_{C}(t) = V_{kilde} \cdot e^{-\frac{t}{\tau}}$ under utladning.
        Begge er uttrykt ved tiden $t$ gitt i sekunder, og gjelder for $t > 0$.
        $\tau = R \cdot C$, $R$ er motstand målt i Ohm, og $C$ er kapasitans målt i Fahrrad.
    \end{itemize}

\subsection{Teori del 2.}

    \begin{itemize}
        \item[-] PMOS-, og NMOS-transistor: De mest grunnleggende byggeklossene i logiske kretser er disse to formene for transistor, som begge fungerer som brytere.
        PMOS slipper signaler gjennom ved input: lav, og blokkerer dem ved input: høy.
        NMOS fungerer på samme måte, men invertert.
        Se figur~\ref{fig:pnmos_transistors} Ved bruk av disse kan man konstruere alle logiske porter.
        \item[-] NAND port: En to-inngangs NAND port er en logisk port som tar to binære input signaler og utfører funksjonene NOT og AND på dem for å produsere ett output signal.
        Se figur~\ref{fig:tt_nand} for sannhetstabellen til NAND porten.
        Se figur~\ref{fig:NAND_transistorlevel} for oppbyggingen av en NAND port med bruk av PMOS-, og NMOS-transistorer.
    \end{itemize}

    \begin{figure}[!htb]
        \centering
        \includegraphics[height=6cm]{figurer/PNMOS.png}
        \caption{PMOS og NMOS- transistor med deres ideelle ekvivalenter for åpen og lukketkanal.}
        \label{fig:pnmos_transistors}
    \end{figure}

    \begin{figure}[!htb]
        \centering
        \includegraphics[height=6cm]{figurer/NAND_transistorlevel.png}
        \caption{NAND port på transistornivå.}
        \label{fig:NAND_transistorlevel}
    \end{figure}

    \begin{figure}[!htb]
        \centering
        \caption{Sannhetstabell for NAND port}
        \label{fig:tt_nand}
        \begin{tabular}{|c|c|c|}
            \hline
            \textbf{A} & \textbf{B} & \textbf{A NAND B} \\ \hline
            0 & 0 & 1 \\
            0 & 1 & 1 \\
            1 & 0 & 1 \\
            1 & 1 & 0 \\ \hline
        \end{tabular}
    \end{figure}


\clearpage

\section{Gjennomføring}

Gjennomføringen deles i to hoveddeler, RC-Krets og CMOS logikk. De to arbeidene er uavhengige av hverandre.\\
Ved porter refereres det til inngangene på veroboard sokkel, som samsvarer med de nederst på kretskort figuren i appendiks 1

\subsection{RC-Krets}

\subsubsection{Lodding}
Først ble kretsen, som beskrevet i Figur~\ref{fig:krets_3_3}, loddet på kretskortet. En 1k$\Omega$ motstand ble loddet på posisjon $R_{10}$, og en 15nF kondensator ble loddet på posisjon $C_{1}$

\begin{figure}[!htb]
    \centering
    \begin{circuitikz}
        \draw
            (0,0) to [vsourcesquare] (0,6)
            to [european resistor] (12,6)
            to [capacitor] (12,0) -- (0,0)
        ;
        \draw[fill] (0,6) circle [radius=0.1];
        \draw[fill] (12,6) circle [radius=0.1];
        \draw[fill] (12,0) circle [radius=0.1];
        \draw[fill] (0,0) circle [radius=0.1];
        \node [above] at (6,5) {$R_{10}=1k\Omega$};
        \node [above left] at (12,2) {$C_{1}=1nF$};
        \node [above left] at (0, 2) {$Sig.gen=(0,4)V$};
    \end{circuitikz}
    \caption{RC Krets}
    \label{fig:krets_3_3}
\end{figure}

\subsubsection{Signalgenerator}
Signalgeneratoren ble stilt inn slik at den sendte ut en firkantpuls med frekvens på 3 KHz og amplitude på 4V, med 2V i offset slik at signalet har toppunkt i 4V og bunnpunkt i 0V.

\subsubsection{Måling}
Signalgeneratoren, ved hjelp av et T-ledd, ble så koblet til både kretsen via veroboardet (port 19 og port 17), og til oscilloskopet. Ved å måle spenningen over kondensatoren vc (port 18 og port 17) og spenningen fra generatoren på forskjellige kanaler var det mulig å se både spenningen som signalgeneratoren påtrykte, og hvordan spenningen endret seg etter å ha gått gjennom kondensatoren.

\subsubsection{Sitering}
I dette dokumentet har jeg valgt å bruke Biblatex. Det er flere forskjellige pakker tilgjengelig for
kildehenvisning, men det er denne jeg er mest kjent med. For å sitere en kilde er dere nødt til å gjøre 4 ting:
\begin{enumerate}
    \item definere referansestil og referansebibliotekfil (gjort i setup.tex).
    \item lage selve kilden i en .bib-fil.
    \item referere ved å benytte cite-kommandoen.
    \item printe referanseliste med printbibliography-kommandoen (gjort i main.tex).
\end{enumerate}

I mylib.bib finner dere noen eksempler på kilder. Ellers kan jeg anbefale dokumentasjonen for biblatex \cite{biblatex}.
Nå som vi har referert til en kilde legges denne til i lista. Der havner de i rekkefølgen de ble referert til.
Legg spesielt merke til hvordan man kun kildene som faktisk referes til vises. 

%Hvis dere fjerner kommentaren på denne kilden~\cite{airsim_ikke_i_bruk}, vil den også vises i lista.

\newpage % bruk denne for å starte en ny side.
\subsection{CMOS logikk}

\subsubsection{Lodding}
Først ble kretsen, som beskrevet i figur 2 (3-3), loddet opp på kretskortet, med verdier som anvist i figuren. I tillegg måtte strappingen S2 loddes inn for koblet sammen logikk kretsen og driver kretsen.


\begin{figure}[!htb]
    \centering
    \begin{circuitikz}
        \draw (0,4) node[circ, scale=1.5] (A){A};
        \draw (0,2) node[circ, scale=1.5] (B){B};
        \draw (0,0) node[circ, scale=1.5] (C){C};

        \draw (2, -4) node[genericshape, rotate=90] (R4){$R_4$};

        \draw (2,0) node[circ, scale=1.5] (R4P){};
        \draw (4,2) node[circ, scale=1.5] (R3P){};
        \draw (6,4) node[circ, scale=1.5] (R2P){};

        \draw (10,1) node[and port] (and1){};
        \draw (12,3) node[and port] (and2){};

        \draw (C) -| (and1.in 2);
        \draw (B) -| (and1.in 1);
        \draw (A) -| (and2.in 1);
        \draw (R4) -- (R4P);
        \draw (and1.out) -| (and2.in 2);
    \end{circuitikz}
    \caption{Logisk Krets}
    \label{fig:krets_3_3}
\end{figure}

\subsubsection{Måling}
Det ble så målt spenningene til nodene D og Q på figur 2 med forskjellige inn-signaler på A, B, og C. 
Spenningene ble målt i fra jord (VSS). For å måle spenningen til node D ble proben til oscilloskop satt på port 9 (VSS) og port 4 på veroboard sokkelen. 
For node Q ble port 9 og port 5 brukt.



\subsection{tikz}
Tikz er et veldig kraftig tegneverktøy til Latex, og denne innføringen skraper bare såvidt på overfraten.
I denne pakken er det så mange muligheter at jeg ikke kommer til å skrive noe detaljert rundt det.
For en introduksjon til grunnpakken av tikz, kan dere se på Jaques Crémer sin introduksjon \cite{tikz_intro}.

 viser en enkel kretstegning tegnet med circuitikz-pakken, mens figur~\ref{fig:arb_3D_fig} er vesentlig mye mer avansert, og benytter seg av tikz-3dplots-pakken.


% --------------------------------------------------------------------------
% Dette er kun for de som er interessert i mer avansert figurlaging med Tikz
% --------------------------------------------------------------------------

% \begin{figure}[!htb]
%     \centering
%     \tdplotsetmaincoords{70}{100}
%     \begin{tikzpicture}[tdplot_main_coords, scale = 0.7]
%         \coordinate (O) at (0,0,0);
        
%         \pgfmathsetmacro{\rvec}{4}
%         \pgfmathsetmacro{\thetavec}{120}
%         \pgfmathsetmacro{\phivec}{50}
        
%         \tdplotsetcoord{P}{\rvec}{\phivec}{\thetavec}
        
%         \draw[thick, ->] (-3,0,0) -- (5,0,0) node [below left]{$x$};
%         \draw[thick, ->] (0,-1,0) -- (0,5,0) node [below right]{$y$};
%         \draw[thick, ->] (0,0,-1) -- (0,0,5) node [below left]{$z$};
        
%         \draw[->, thick, red] (O) -- (P);
%         \draw[] (P) node[above right]{$P$};
%         \node at (P){\textbullet};
%         \draw[dashed, thick, red] (O) -- (Pxy);
%         \draw[dashed, thick, blue] (Pxy) -- (0,2.654,0); %rsin(theta)sin(phi)
%         \draw[dashed, thick, blue] (Pxy) -- (-1.286,0,0); %rcos(theta)sin(phi)
%         \draw[dashed, thick, green] (P) -- (0,0,2.571); %rcos(phi)
        
%         \tdplotdrawarc{(O)}{1}{0}{\thetavec}{anchor=north}{$\theta$}
%         \tdplotsetthetaplanecoords{\thetavec}
%         \tdplotdrawarc[tdplot_rotated_coords]{(O)}{1}{0}{\phivec}{anchor=south west}{$\phi$}
        
%         \foreach \angle in {0,60,...,300} {
%             \tdplotsetthetaplanecoords{\angle}
%             \ifthenelse{\angle=\thetavec}{
%                 \draw[tdplot_rotated_coords] (\rvec,0,0) arc (0:180:\rvec);
%             }{
%                 \ifthenelse{\angle<90 \OR \angle>270}{
%                     \draw[tdplot_rotated_coords] (\rvec,0,0) arc (0:180:\rvec);
%                 }{
%                     \draw[dash pattern=on 3pt off 10pt,tdplot_rotated_coords] (\rvec,0,0) arc (0:180:\rvec);
%                 }
%             }
%         } % end \foreach
        
%         \draw[] (0,-\rvec,0) arc (-90:120:\rvec);
%         \draw[dashed] (120:\rvec) arc (120:270:\rvec);
        
%         \foreach \angle in {-60,-59.5,...,0} {
%             \tdplotsetthetaplanecoords{\angle}
%             \draw[tdplot_rotated_coords, color = gray!60, opacity = 0.7] (0:\rvec) arc (0:180:\rvec);
%         } % end foreach
        
%         \shade[ball color = blue!40, opacity = 0.4] (0,0) circle (4cm); %cm does fix things :O
        
%         \tdplotsetrotatedcoordsorigin{(P)}
%         \tdplotsetrotatedcoords{0}{-90}{90}
%         \draw[tdplot_rotated_coords,->] (-1,0,0) -- (3,0,0) node [below right]{$x'$};
%         \draw[tdplot_rotated_coords,->] (0,-1,0) -- (0,3,0) node [right]{$y'$};
%         \draw[tdplot_rotated_coords,->] (0,0,-1) -- (0,0,3) node [above right]{$z'$};
    
%     \end{tikzpicture}
    
    
%     \caption{Avansert 3D-figur}
%     \label{fig:arb_3D_fig}
% \end{figure}\clearpage

\section{Resultater}

Resultatene fra begge forsøkene ble skrevet ned i labjournalen før den ble digitalisert

\subsection{RC-Krets}

Som Figur~\ref{fig:rc-resultater} viser hang spenningen over kondensatoren etter spenningen fra signalgeneratoren 

\begin{figure}[!htb]
    \centering
    \includegraphics[height=6cm]{figurer/RC-figur.pdf}
    \caption{Spenningen over kondensatoren og signalgeneratoren}
    \label{fig:rc-resultater}
\end{figure}

Tisdkonstanten $\tau$ ble målt til å være $17\cdot 10^{-6}s$ ved prosedyren forklart i \ref{subsec:rc-maaling}

\subsection{CMOS logikk}

Figur~\ref{fig:cmos-resultater} viser spenningene målt ved punktene D og Q fra CMOS logikk kretsen som vist i Figur~\ref{fig:rc}. Målingene av D og Q er gjort i Volt.

\begin{figure}[!htb]
    \centering
    \begin{tabular}{|c|c|c|c|c|}
        \hline
        \textbf{A} & \textbf{B} & \textbf{C} & \textbf{D} & \textbf{Q} \\ \hline
        0 & 0 & 0 & 9 & 8.4 \\
        1 & 0 & 0 & 9 & 0 \\
        1 & 1 & 0 & 9 & 0 \\
        1 & 0 & 1 & 9 & 0 \\
        1 & 1 & 1 & 0 & 0 \\
        0 & 0 & 1 & 9 & 8.4 \\
        0 & 1 & 0 & 9 & 8.4 \\
        0 & 1 & 1 & 0 & 8.7 \\ \hline
    \end{tabular}
    \caption{Målingene av CMOS spenningene}
    \label{fig:cmos-resultater}
\end{figure}\clearpage

\section{Diskusjon}

\subsection{RC-Krets}

    Tau ble målt til $17 \cdot 10^{-6} \Omega F$, målingen viser et avvik fra det teoretiske $15 \cdot 10^{-6} \Omega F$ på ca $13.4$ prosent.
    Dette er ikke oppsiktsvekkende, da målemetoden vår delvis baserte seg på et øyemål på oscilloskopet.
    Videre avvikskilder kan svare til avvik i verdier fra komponentspesifikasjonene (se appendiks G) og motstand i ledninger ol.
    Som forventet av teorien viser målingene$^{\ref{fig:rc-resultater}}$ at spenningen over kondensatoren ikke direkte følger spenningskilden, men er en demping av den over tid.
    Kondensatoren virker altså som en motstand mot endring i spenningen i kretsen.
    Konsekvensene av dette er mange, kondensatorer kan for eksempel brukes til å filtrere høyfrekvent støy i et analogt signal.
    Ettersom kondensatorer motstår momentane endringer i spenning vil høyfrekvente spenningsendringer kun føre til små endringer i den totale spenningen.
    Et annet bruksområde er konvertering mellom AC-, og DC-strøm.
    Vår enkle RC krets kan sees på som at den ‘skviser’ firkantpulssignalet sammen mot likevektslinjen, i dette tilfellet to volt, og med en høyere kapasitans (eller motstand) vil denne ‘skvisefaktoren’ blitt sterkere, og trende spenningen mot likevektslinjen.
    Vi kan tenke oss at dersom RC går mot uendelig går spenningen mot v0. Dette vil medføre at kretsen fjerner all variasjon fra utgangstillstanden.

\subsection{CMOS logikk}

    Resultatene av målingene av den kombinatoriske kretsen var som forventet, men det var en måling som stakk seg ut.
    Port Q er en NAND port og gav derfor output HØY når èn eller begge av inngangene stod på HØY.
    Allikevel måltes spenningen over Q høyere i det tilfellet der begge inngangene stod på LAV$^{\ref{fig:cmos-resultater}}$.
    Porter har intern motstand i transistorene de består av.
    Ved å analysere NMOS-transistorene i NAND-porten$^{\ref{fig:NAND_transistorlevel}}$ som motstander finner vi forklaringen på avviket.
    Når kun ett av inngangssignalene står LAV går strømmen kun gjennom èn av NMOS-transistorene i porten, som kan sees ekvivalent med at spenningen fra driverkretsen (9V) påvirkes av en (liten) motstand.
    I det tilfelle der begge står LAV vil strømmen kunne gå gjennom begge NMOS-transistorene.
    Vi vet fra resultatene i kapittel 3.2 i \cite{ECandDD} at motstander i parallell har mindre motstand enn den minste enkelt-motstanden, det er derfor naturlig at når begge inngangssignalene er LAV vil det være mindre spenningsfall, og derav høyere spenning over Q.

    På bakgrunn av dette oppstår det et problem med å tolke logiske kretser som rene modeller der man anser input og output til å være binære.
    Derfor oppgis det intervaller for logisk høy og logisk lav i spesifikasjoner for logiske porter.
    I vårt tilfelle var disse intervallene ca.\ $[-0.5V,\ 3V]$ for logisk lav og $[7V,\ 9.5V]$ for logisk høy (se appendiks G).
    Ca.\ da vi hadde $vdd=9V$ og ikke $10V$ som spesifisert i databladet.

    Det at en logisk krets på lavt nivå består av elektriske komponenter er det som medfører at signaler påvirkes av eksternt støy og interne faktorer.
    Dette viser til at det i virkeligheten ikke finnes digitale signaler.
    Men vi tillnærmer, med de unøyaktighetene som kan oppstå.\clearpage

\printbibliography
\addcontentsline{toc}{section}{Referanser}
\clearpage

\includepdf[pages=-,pagecommand=\thispagestyle{plain}]{appendixL.pdf}

\end{document}
