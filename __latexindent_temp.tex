\section{Gjennomføring}

Gjennomføringen deles i to hoveddeler, RC-Krets og CMOS logikk. De to arbeidene er

\subsection{RC-Krets}

I dette dokumentet har jeg valgt å bruke Biblatex. Det er flere forskjellige pakker tilgjengelig for
kildehenvisning, men det er denne jeg er mest kjent med. For å sitere en kilde er dere nødt til å gjøre 4 ting:
\begin{enumerate}
    \item definere referansestil og referansebibliotekfil (gjort i setup.tex).
    \item lage selve kilden i en .bib-fil.
    \item referere ved å benytte cite-kommandoen.
    \item printe referanseliste med printbibliography-kommandoen (gjort i main.tex).
\end{enumerate}

I mylib.bib finner dere noen eksempler på kilder. Ellers kan jeg anbefale dokumentasjonen for biblatex \cite{biblatex}.
Nå som vi har referert til en kilde legges denne til i lista. Der havner de i rekkefølgen de ble referert til.
Legg spesielt merke til hvordan man kun kildene som faktisk referes til vises. 

%Hvis dere fjerner kommentaren på denne kilden~\cite{airsim_ikke_i_bruk}, vil den også vises i lista.

\newpage % bruk denne for å starte en ny side.
\subsection{Egendefinerte figurer i Tikz og kildehenvisning}

Tikz er et veldig kraftig tegneverktøy til Latex, og denne innføringen skraper bare såvidt på overfraten.
I denne pakken er det så mange muligheter at jeg ikke kommer til å skrive noe detaljert rundt det.
For en introduksjon til grunnpakken av tikz, kan dere se på Jaques Crémer sin introduksjon \cite{tikz_intro}.

Figur~\ref{fig:arb_enkel_krets} viser en enkel kretstegning tegnet med circuitikz-pakken, mens figur~\ref{fig:arb_3D_fig} er vesentlig mye mer avansert, og benytter seg av tikz-3dplots-pakken.

\begin{figure}[!htb]
    \centering
    \begin{circuitikz}
        % \draw (x,y) <type>[<options>] (<referansenavn>){<tekst>}
        % Terminer hver kommando med semikolon;
        \draw (0,2) node[xor port] (xor1){};
        \draw (0,0) node[and port] (and1){};
        \draw (2,1) node[nand port] (nand1){};
        
        % \node tegner node direkte
        % \node (<navnereferanse>) [<options>]{tekst}
        \node (utg) [right of = nand1, xshift=-0.8cm, yshift=0.3cm]{Q};
        \node (utg) [left of = xor1, xshift=-0.8cm, yshift=0.3cm]{A};
        \node (utg) [left of = xor1, xshift=-0.8cm, yshift=-0.3cm]{B};
        \node (utg) [left of = and1, xshift=-0.8cm, yshift=0.3cm]{C};
        \node (utg) [left of = and1, xshift=-0.8cm, yshift=-0.3cm]{D};
        
        \draw (xor1.out) -| (nand1.in 1);
        \draw (and1.out) -| (nand1.in 2);
    \end{circuitikz}
    \caption{Enkel krets}
    \label{fig:arb_enkel_krets}
\end{figure}

% --------------------------------------------------------------------------
% Dette er kun for de som er interessert i mer avansert figurlaging med Tikz
% --------------------------------------------------------------------------

\begin{figure}[!htb]
    \centering
    \tdplotsetmaincoords{70}{100}
    \begin{tikzpicture}[tdplot_main_coords, scale = 0.7]
        \coordinate (O) at (0,0,0);
        
        \pgfmathsetmacro{\rvec}{4}
        \pgfmathsetmacro{\thetavec}{120}
        \pgfmathsetmacro{\phivec}{50}
        
        \tdplotsetcoord{P}{\rvec}{\phivec}{\thetavec}
        
        \draw[thick, ->] (-3,0,0) -- (5,0,0) node [below left]{$x$};
        \draw[thick, ->] (0,-1,0) -- (0,5,0) node [below right]{$y$};
        \draw[thick, ->] (0,0,-1) -- (0,0,5) node [below left]{$z$};
        
        \draw[->, thick, red] (O) -- (P);
        \draw[] (P) node[above right]{$P$};
        \node at (P){\textbullet};
        \draw[dashed, thick, red] (O) -- (Pxy);
        \draw[dashed, thick, blue] (Pxy) -- (0,2.654,0); %rsin(theta)sin(phi)
        \draw[dashed, thick, blue] (Pxy) -- (-1.286,0,0); %rcos(theta)sin(phi)
        \draw[dashed, thick, green] (P) -- (0,0,2.571); %rcos(phi)
        
        \tdplotdrawarc{(O)}{1}{0}{\thetavec}{anchor=north}{$\theta$}
        \tdplotsetthetaplanecoords{\thetavec}
        \tdplotdrawarc[tdplot_rotated_coords]{(O)}{1}{0}{\phivec}{anchor=south west}{$\phi$}
        
        \foreach \angle in {0,60,...,300} {
            \tdplotsetthetaplanecoords{\angle}
            \ifthenelse{\angle=\thetavec}{
                \draw[tdplot_rotated_coords] (\rvec,0,0) arc (0:180:\rvec);
            }{
                \ifthenelse{\angle<90 \OR \angle>270}{
                    \draw[tdplot_rotated_coords] (\rvec,0,0) arc (0:180:\rvec);
                }{
                    \draw[dash pattern=on 3pt off 10pt,tdplot_rotated_coords] (\rvec,0,0) arc (0:180:\rvec);
                }
            }
        } % end \foreach
        
        \draw[] (0,-\rvec,0) arc (-90:120:\rvec);
        \draw[dashed] (120:\rvec) arc (120:270:\rvec);
        
        \foreach \angle in {-60,-59.5,...,0} {
            \tdplotsetthetaplanecoords{\angle}
            \draw[tdplot_rotated_coords, color = gray!60, opacity = 0.7] (0:\rvec) arc (0:180:\rvec);
        } % end foreach
        
        \shade[ball color = blue!40, opacity = 0.4] (0,0) circle (4cm); %cm does fix things :O
        
        \tdplotsetrotatedcoordsorigin{(P)}
        \tdplotsetrotatedcoords{0}{-90}{90}
        \draw[tdplot_rotated_coords,->] (-1,0,0) -- (3,0,0) node [below right]{$x'$};
        \draw[tdplot_rotated_coords,->] (0,-1,0) -- (0,3,0) node [right]{$y'$};
        \draw[tdplot_rotated_coords,->] (0,0,-1) -- (0,0,3) node [above right]{$z'$};
    
    \end{tikzpicture}
    
    
    \caption{Avansert 3D-figur}
    \label{fig:arb_3D_fig}
\end{figure}