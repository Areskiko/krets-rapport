\section{Diskusjon}

\subsection{RC-Krets}

    Tau ble målt til $17 \cdot 10^{-6} \Omega F$, målingen viser et avvik fra det teoretiske $15 \cdot 10^{-6} \Omega F$ på ca $13.4$ prosent.
    Dette er ikke oppsiktsvekkende, da målemetoden vår delvis baserte seg på et øyemål på oscilloskopet.
    Videre avvikskilder kan svare til avvik i verdier fra komponentspesifikasjonene (se appendiks G) og motstand i ledninger ol.
    Som forventet av teorien viser målingene$^{\ref{fig:rc-resultater}}$ at spenningen over kondensatoren ikke direkte følger spenningskilden, men er en demping av den over tid.
    Kondensatoren virker altså som en motstand mot endring i spenningen i kretsen.
    Konsekvensene av dette er mange, kondensatorer kan for eksempel brukes til å filtrere høyfrekvent støy i et analogt signal.
    Ettersom kondensatorer motstår momentane endringer i spenning vil høyfrekvente spenningsendringer kun føre til små endringer i den totale spenningen.
    Et annet bruksområde er konvertering mellom AC-, og DC-strøm.
    Vår enkle RC krets kan sees på som at den ‘skviser’ firkantpulssignalet sammen mot likevektslinjen, i dette tilfellet to volt, og med en høyere kapasitans (eller motstand) vil denne ‘skvisefaktoren’ blitt sterkere, og trende spenningen mot likevektslinjen.
    Vi kan tenke oss at dersom RC går mot uendelig går spenningen mot v0. Dette vil medføre at kretsen fjerner all variasjon fra utgangstillstanden.

\subsection{CMOS logikk}

    Resultatene av målingene av den kombinatoriske kretsen var som forventet, men det var en måling som stakk seg ut.
    Port Q er en NAND port og gav derfor output HØY når èn eller begge av inngangene stod på HØY.
    Allikevel måltes spenningen over Q høyere i det tilfellet der begge inngangene stod på LAV$^{\ref{fig:cmos-resultater}}$.
    Porter har intern motstand i transistorene de består av.
    Ved å analysere NMOS-transistorene i NAND-porten$^{\ref{fig:NAND_transistorlevel}}$ som motstander finner vi forklaringen på avviket.
    Når kun ett av inngangssignalene står LAV går strømmen kun gjennom èn av NMOS-transistorene i porten, som kan sees ekvivalent med at spenningen fra driverkretsen (9V) påvirkes av en (liten) motstand.
    I det tilfelle der begge står LAV vil strømmen kunne gå gjennom begge NMOS-transistorene.
    Vi vet fra resultatene i kapittel 3.2 i \cite{ECandDD} at motstander i parallell har mindre motstand enn den minste enkelt-motstanden, det er derfor naturlig at når begge inngangssignalene er LAV vil det være mindre spenningsfall, og derav høyere spenning over Q.

    På bakgrunn av dette oppstår det et problem med å tolke logiske kretser som rene modeller der man anser input og output til å være binære.
    Derfor oppgis det intervaller for logisk høy og logisk lav i spesifikasjoner for logiske porter.
    I vårt tilfelle var disse intervallene ca.\ $[-0.5V,\ 3V]$ for logisk lav og $[7V,\ 9.5V]$ for logisk høy (se appendiks G).
    Ca.\ da vi hadde $vdd=9V$ og ikke $10V$ som spesifisert i databladet.

    Det at en logisk krets på lavt nivå består av elektriske komponenter er det som medfører at signaler påvirkes av eksternt støy og interne faktorer.
    Dette viser til at det i virkeligheten ikke finnes digitale signaler.
    Men vi tillnærmer, med de unøyaktighetene som kan oppstå.